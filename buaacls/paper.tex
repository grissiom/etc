%!TEX TS-program = xelatex
% XeLateX
\input{ctex4xetex.cfg}


\documentclass[xelatex]{BUAApaper}


% XeLateX
\XeTeXdefaultencoding "utf-8"
\usepackage[BoldFont, SlantFont, CJKtextspaces]{xeCJK}
\punctstyle{kaiming}
%\setmainfont{DejaVu Sans}
\setCJKmainfont{宋体}
\setCJKmonofont{文泉驿正黑}


\bibliographystyle{BUAAbibsty}
%\bibliographystyle{unsrt}
%\input paper_preamble.tex


% XeLateX
\usepackage{fontspec}%No options for fontspec
\usepackage{xunicode}% provides unicode character macros
\usepackage{xltxtra} % provides some fixes/extras
\defaultfontfeatures{Mapping=tex-text}


\begin{document}
\author{张三}
\eauthor{Zhang San}
\title{远程内存池的设计与实现}
\etitle{Remote Memory Pool---Design and Implementation}
\date{200X年6月}
\advisor{王小五}{Wang Xiaowu}
\schoolname{计算机学院}
\specialty{计算机科学与技术}
\unitcode{由教务处统一填写}
\studentnumber{3XXXXXXX}
\classification{TP301}
\confidential{不涉密}

\maketitle
\buaakeyword{第一, 第二}{First, Second}
\setcounter{page}{0}


\frontmatter
\newpage
\begingroup
\hrule width0bp
\vskip54bp
\centerline{\bf\zihao{-2}本人声明}
\vskip27bp
\zihao{4}我声明,本论文及其研究工作是由本人在导师指导下独立完成的,在完成论文时所利用的一
切资料均已在参考文献中列出。
\vskip63bp
\hfill	\begin{tabular}{cl}
		作者:&张三\\
		签字:&\vrule width0pt height10bp depth10bp{}\\
		时间:& 200X年Y月
	\end{tabular}
\endgroup
\begin{cabstract}

随着网络速度的提高,集群系统的应用越来越普遍,如何有效利用集群中空闲结点的内存以
提高集群的整体性能这一问题越来越受到重视。

本文介绍了一种...

本文实现了一个~XXX~的原形系统,并对它的可用性和性能进行了测试。测试表明...

本文将首先对近年来相关的研究进行叙述,指出前人已经进行的工作和问题所在;之后,本
文将介绍远程内存池的设计和远程内存池原形系统的实现;再后将给出对原形系统的测试数
据,说明它的可用性和性能;最后将指出不足和改进的方向。

\end{cabstract}

\begin{eabstract}

In recent years there has been a great interest in the use of clusters.
Exploiting idle memory in clusters to avoid swapping and improve the performance
is taken more and more attention.

In this paper, we introduce ...

This paper describes the design and implementation of XXX, and also implements
a prototype. We tested and profiled the prototype, and found that ...

This paper first introduce previous work in related fields, then present the
design detail of our implementation. In section 5, it deals with the detailed
performance evaluation and analysis of our designs. Then, in the last section,
it draw the conclusion and state future works.

\end{eabstract}



\tableofcontents
\newpage
\mainmatter

\section{绪论}

\subsection{背景及目的}
介绍背景及目的.

\subsection{国内外相关研究}
介绍国内外相关研究.

根据要求, 如果引用是句子的一部分, 如: \cite{sigsegv}提出, ...,
就要用\verb+\cite+; 否则, 如: 有些人认为不是这样的\upcite{idlemem, Myrinet,
ulk3rd}. 但是..., 就要用\verb+\upcite+

\subsection{问题的提出}
提出问题.

\subsection{论文构成}
论文结构.

 %绪论
\section{技术背景}

\subsection{Linux~操作系统}

\subsection{内核模块}

\subsection{地址空间与~VMA}
 %背景
\section{结构与流程}

这一部分保密. 

图\ref{figure:example3}中,....

\begin{figure}[htb!]
\begin{center}
\small
%\input figure_example3.tex
\end{center}
%\caption{示例三\label{figure:example3}}
\end{figure}
 %设计与实现
\section{设计与实现}

这一部分保密.

 %关键算法与协议

\section{性能测试}

性能很好啊.

 %测试结果
\section*{结论}\addcontentsline{toc}{section}{结论}

根据要求, 结论一节是没有节号的.

 %总结

\newpage
\renewcommand{\thesection}{致谢}
\section*{致谢}\addcontentsline{toc}{section}{致谢}

首先,我要衷心感谢北京航空航天大学计算机学院四年来对我的辛勤培养。我在计算机学院
四年的经历将成为我一生的宝贵财富。

然后, 我要感谢....

感谢~Linus Torvalds,他无私的开放了~Linux~的源代码,为我们提供了课题和饭碗。感谢
~VIM~的作者~Bram Moolenaar(虽然我没有给乌干达的可怜儿童和~Bram~捐款),他设计的编
辑器是世界上最好的编辑器。感谢 ~\TeX{}~ 的作者 ~D. E. Knuth~ 教授、
~\LaTeXe{} 的作者~Leslie Lamport~以及 ~\LaTeXe{}~ 中无数个宏包的设计者们,他们的设
计组成的这套排版系统使我可以排出精美的论文,另外在紧张的论文撰写同时给我带来了许
多快乐。

最后,感谢~\CTeX~社区的所有朋友们。

\nocite{*}
%\standardtilde
\newpage
\bibliography{bibs.bib}\addcontentsline{toc}{section}{参考文献}
\appendix
%\section{XXX~源码简介}

本附录对~XXX~的源代码做一简要介绍。

以下是~XXX~源码树的结构:

\begin{sourcelist}\begin{verbatim}
|-- Makefile
|-- common
|   |-- Makefile
|   |-- config.h
|   |-- debug.c
|   |-- debug.h
|   |-- fast_printk.c
|   |-- slabs.c
|   |-- slabs.h
|   |-- tty_printk.c
|   `-- ulist.h
\end{verbatim}\end{sourcelist}



%\section{XXX~关键数据结构源码}

本附录列出~RMP~关键数据结构的代码,以便读者理解前文提及的数据结构。

\subsection{connector}

connector.h~文件定义了~|connector|~结构。

\lstset{numbers=left, numberstyle=\tiny, stepnumber=1, numbersep=5pt,
	texcl=true, escapechar={@}}
\begin{sourcelist}\begin{lstlisting}
/*
 *
@ * connector.h @
@ * @
@ * ZhangSan, Mar. 20, 200X @
@ * @
@ * 连接抽象层是一个面向内核提供抽象连接的机制,每种类型的连接通过提供 @
@ * 一组~ops~和一个连接标识向核心应用提供透明的连接. @
 */

#ifndef CONNECTOR_H
#define CONNECTOR_H

#include "common/config.h"

#ifdef __KERNEL__
#include <linux/module.h>
#include <linux/byteorder/generic.h>
#include <asm/msr.h>
#endif

#define CONNSTRING_MAX 32
#define CONN_TYPE_SOCKET 1
#define CONN_TYPE_NOEXIST 10

@/* 消息标识 */@

\end{lstlisting}\end{sourcelist}


%\section{XXX~的使用}
\lstset{numbers=none}

本附录介绍~XXX~的编译和使用。

\subsection{编译}

XXX~的源码经测试可以在~Kernel 2.6.11.12~和~Kernel 2.6.13~上编译通过...

获得源代码后,在根目录下执行``make''即可开始编译。正常的编译过程将有以下输出:
\begin{sourcelist}\begin{verbatim}
$ make
make -C manager
make[1]: Entering directory `/tmp/rmp/manager'
rm -f ../common/debug.o
\end{verbatim}\end{sourcelist}

之后会生成以下文件:
\begin{sourcelist}\begin{verbatim}
./common/fast_printk.ko
./common/tty_printk.ko
\end{verbatim}\end{sourcelist}


\clearpage
\end{document}

