\section{XXX~的使用}
\lstset{numbers=none}

本附录介绍~XXX~的编译和使用。

\subsection{编译}

XXX~的源码经测试可以在~Kernel 2.6.11.12~和~Kernel 2.6.13~上编译通过...

获得源代码后,在根目录下执行``make''即可开始编译。正常的编译过程将有以下输出:
\begin{sourcelist}\begin{verbatim}
$ make
make -C manager
make[1]: Entering directory `/tmp/rmp/manager'
rm -f ../common/debug.o
\end{verbatim}\end{sourcelist}

之后会生成以下文件:
\begin{sourcelist}\begin{verbatim}
./common/fast_printk.ko
./common/tty_printk.ko
\end{verbatim}\end{sourcelist}

